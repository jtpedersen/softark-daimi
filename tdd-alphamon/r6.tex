\documentclass[a4paper]{article}
\usepackage[utf8]{inputenc}
\usepackage[danish]{babel}
%\usepackage[T1]{fontenc}
\usepackage{amsmath,amssymb}
\usepackage{fancyhdr}
\usepackage{booktabs}
\usepackage{graphicx}
\pagestyle{fancy}

%\usepackage{tabular}

\begin{document}
\chead{dSoftArk 2007 -- group 20 -- uge 6}
\lhead{}
\rhead{}

\section{Equivalence partitions}

\begin{itemize}
\item
\begin{verbatim}
d(f,t) != die
\end{verbatim}


\item
\begin{verbatim}
getCount(from)>0 && getColor(from) == playerInTurn
\end{verbatim}


\item
\begin{verbatim}
to != blocked
\end{verbatim}

\end{itemize}

\subsection{partitioning of the input space}
validity = v ( move, player, board-state, die-value ) 

distance-function', d(f,t) = distance between points 'f' and 't'

count function :num(p) = number og checkers at point p

color function :col(p) = color of checker at point p

player const: oPlayer = other player 

\subsection{Standardmoves}

\noindent
\begin{tabular}{l|c|c}
Dimension & Invalid Partition & Valid Partition \\
\toprule
move(f,t) & f,t $\in$ \{bearOff,bar\} [1] & f,t $\notin$ \{bearOff,bar\} [6] \\
\midrule
die-value & d(f,t)!=die [2] & d(f,t)== die [7] \\
\midrule
boardState & num(bar) $\neq$ 0 [3];    & ( col(f) = player ) \\
           & col(f) $=$ oPlayer [4]; col(f) $=$ none [9];
              & num(t) $<$ 2 $\vee$ \\
           & num(t)$>$1 $\vee$ col(t) $=$ oPlayer [7] & col(t)=oPlayer
           [8]; \\
           & & num(t) = 0 [12]\\
       &   & col(t) $=$ player [13] \\
\midrule
player & - & player $=$ black [10]; \\
       &   & player $=$ red [11] \\

\bottomrule
\end{tabular}

\subsection{Test cases}
\label{1:testcases}

\begin{tabular}{c|c|c}
AEkvialensklasse        & Test                      & Output    \\
\toprule
$[1]$                   & $f=bar$                   & false     \\
\midrule
$[2]$                   & $f=R1,\ t=R3,$            & false     \\
                        & $dice=5$                  &           \\
\midrule
$[3]$                   & $num(bar)=3$              & false     \\
\midrule
$[4]$                   & $col(R1)=BLACK,$          & false     \\
                        & $player=RED,\ f=R1$       &           \\
\midrule
$[5]$                   & $col(R6)=Red,\ num(R6)=6$ & false     \\
                        & $player=B,\ t=R6$         &           \\
\midrule
$[6]\times[7]\times[8]\times[10]$ & $f=R1,\ t=R2,\ dice=1,$   & true      \\
                        & $player=BLACK,$           &           \\
                        & $num(R2)=1,col(R2)=RED$   &           \\
\midrule
$[6]\times[7]\times[12]\times[11]$ & $f=B1,\ t=B2,\ dice=1,$   & true      \\
                        & $player=RED,$           &           \\
                        & $num(B2)=0$             &           \\
\midrule
$[6]\times[7]\times[13]\times[11]$ & $f=B1,\ t=B2,\ dice=1,$   & true      \\
                        & $player=RED,$           &           \\
                        & $num(B2)=2, col(R2)=RED$             &           \\
\bottomrule
\end{tabular}

\subsection{Coverage og disjointness}
Vi er rimlig sikker paa at have fuldt coverage da vi har 
gaaet alle dimensioner igennem og fundet alle EC. Vi er 
rimlig sikker da vi har brugt systematic testing processen.

Vi mener at have disjunkte EC'er, da vi ikke lige kunne
finde et eksempel der modbeviste det.

\subsection{Boundary Values}
Vi har lavet 2 testcases saa antal brikker af oPlayer paa t location
er baade 1 og 2 som er lige paa og lige under
graensevaridien. Vi kigger paa terningen og felterne som en
maengde, og de har ingen graensevaerdier, da specifikationen ikke
definere hvad der sker for vaerdier uden for maengden.

\section{Bar Moves}
Vi bruger samme tabel fra foer, bare med disse aendringer:

\begin{tabular}{c|c}
EC & Repartionering \\
\toprule
$[1]$ & [1] [1a]; \\
      & t $\in$ \{ bar, bearOff \} , f $\in$ \{ bearOff \}  [1b] \\
\midrule
$[6]$ & [6] [6a]; \\
      & f $\notin$ \{bearOff\}, t $\notin$ \{bar, bearOff\} [6b] \\
\midrule
$[3]$ & [3] [3a]; \\
      & f $\notin$ \{bar\} $\wedge$ num(bar) $>$ 0 [3b] \\
\midrule
$[8]$ & f $\notin$ \{bar\} $\wedge$ "[8]" [8a]; \\
      & f $\in$ \{bar\} $\wedge$ num(bar) $>$ 0 $\wedge$ t $\in$
        \{"opposite innerfield"\} $\wedge$ col(t) = oPlayer $\wedge$
        num(t) =1 [8b] \\
\midrule
$[12]$ & f $\notin$ \{bar\} $\wedge$ "[12]" [12a]; \\
      & f $\in$ \{bar\} $\wedge$ num(bar) $>$ 0 $\wedge$ t $\in$
        \{"opposite innerfield"\} $\wedge$ num(t)=0 [12b] \\
\midrule
$[13]$ & f $\notin$ \{bar\} $\wedge$ "[13]" [13a]; \\
      & f $\in$ \{bar\} $\wedge$ num(bar) $>$ 0 $\wedge$ t $\in$
        \{"opposite innerfield"\} $\wedge$ col(t)=player [13b] \\

\bottomrule
\end{tabular}

\subsection{Test Cases}

Udvidelse fra tabellen fra (\ref{1:testcases}).

\begin{tabular}{c|c|c}
AEkvialensklasse        & Test                      & Output    \\
\toprule
$[1b]$                   & $f=bar$                   & true     \\
\midrule
$[3b]$                   & $f=R1,\ t=R3,$            & false     \\
                        & $dice=2, num(bar)=7$                  &           \\
\midrule
$[6b]\times[8b]\times[11]$  & $f=R\_BAR,\ t=B2,\ dice=2,$   & true      \\
                        & $player=RED,$           &           \\
                        & $num(B2)=2, col(B2)=BLACK$             &
                        \\
\midrule
$[6b]\times[12b]\times[10]$  & $f=B\_BAR,\ t=R2,\ dice=2,$   & true      \\
                        & $player=BLACK,$           &           \\
                        & $num(B2)=0$             &
                        \\
\midrule
$[6b]\times[13b]\times[11]$  & $f=R\_BAR,\ t=B2,\ dice=2,$   & true      \\
                        & $player=RED,$           &           \\
                        & $num(B2)=2, col(B2)=RED$             &
                        \\
\bottomrule
\end{tabular}

\subsection{Coverage og disjointness}
Samme overvejelser som fra Standard Moves.

\subsection{Boundary Values}
Vi har ikke udfoert noget med Bar Moves der aendrede vores boundary values.


\section{Sammenligning}
Vi har glemt at teste for begge farver i naesten alle 
tilfaelde, dette kommer af at vi har kigget paa det som
white-box testing fordi vi brugte tdd processen og saa
det som en triviel implementation.

Vi har de fleste af Standard Moves testene med, selv om
mange af dem er blandet sammen i enkelte testsekvenser.

Ved bar moves har vi ikke testet for moves fra bar til 
hvor man slaar hjem, eller gaar til en plads med egne 
checkers paa.


\end{document}
