\documentclass[a4paper]{article}
\usepackage[utf8]{inputenc}
\usepackage[danish]{babel}
%\usepackage[T1]{fontenc}
\usepackage{amsmath,amssymb}
\usepackage{fancyhdr}
\usepackage[dvipdfm]{graphicx}
\pagestyle{fancy}

\begin{document}
\chead{dSoftArk 2007 -- group 20 -- uge 4}
\lhead{}
\rhead{}


\section*{Exercise 30.8}
\textit{Product: A short explanations of your analysis and design choices.}
Vi lavede en simple extends NullTool og sikrede at der kun hvis der
blev trykket på en terning blev der kaldt next turn.
\begin{verbatim}
   Figure f = objectServer.getEditor().drawing().
      findFigure(e.getX(), e.getY());
    if (f instanceof DieFigure) {
    	objectServer.getGame().nextTurn();
    } 
\end{verbatim}


\section*{Exercise 30.9}
\textit{Product: A short explanations of your analysis and design choices.}
Vi sørgede for at tilføje den nye metode addListener til Game, så vi
havde en liste med gameListernes, og i vores standard board tilføjede vi
to private metoder. NotifyBoardChange og NotifyDiceRolled, som
søgrede for at kalde de relevante metoder på alle de listeners der var
tilføjet. Vi kaldte disse metoder efter bla move og nextTurn.

Vi brugte AlphaMon til som game til udvikle udfra.


\section*{Exercise 30.10}
\textit{Product: A short explanations of your analysis and design choices.}
Vi lavede et MonTool, med et statepattern. Det kan veksle mellem rulle
terninger (diceRollTool) eller flytte brikker(checkerMoveTool) og
sidst et GameWonTool. MonTool satte sig på som gameListener på vores
game og hvergang der blev kaldt tilbage, chekkede vi state, ved at
undersøge om spillet var vundet eller ved at se hvormange moves der
var tilbage. Således kunne vi vælge hvilket tool som MonTool skulle
opføre sig som.



\section*{Exercies 30.11}
Vi lavede to nye ant targets med AlphaMonUpdateState State og
BetaUpdateState, som kunne afvikle nogle moves ved hjælp af en stub
metode, så vi kunne lave et monUserInterface, som tager en appTitle og
en factory i konstruktoren og laver et game.

Slutteligt satte vi det sammen og tilføjede ant targets til *mon, hvor
monUserInterFacer og MonTool bliver brugt. Der med kan man nu spille
*mon med et GUI.




\end{document}
