\documentclass[a4paper]{article}
\usepackage[utf8]{inputenc}
\usepackage[danish]{babel}
%\usepackage[T1]{fontenc}
\usepackage{amsmath,amssymb}
\usepackage{fancyhdr}
\usepackage[dvipdfm]{graphicx}
\pagestyle{fancy}

\begin{document}
\chead{dSoftArk 2007 -- group 20 -- uge 3}
\lhead{}
\rhead{}


\section{Abstract Factory Implementation}
\subsection{Refactoring}
Vi ændrede vores test setup metoder til at tage en factory i constructuren
istedet for vores 2 strategier, og så javac fejle.
Så indførte vi i vores StandardGame så den tog en factory i constructuren
og hentede og satte instansvariabler til vores 2 strategier.
Så indføte vi vores Factory interface, MonFactory med create-metoder til
hver af vores strategier og indførte null-implementationer og så koden
compile og en masse testcases fejle.
Så implementerede vi de rigtige factory klasser for alphamon, betamon og 
gammamon, og alle tests kørte igen.

\section{Deltamon Implementation}
\subsection{Refactoring}
Vi startede med at indse at winner kom til at ændre sig, vi kunne indkredse
dette ansvar til en winner metoder der returnede en Color. Derefter flyttede
vi ansvaret ud i et WinnerStrategy. (3-1-2)
\subsection{Problems}
Vi kom til at tænke på hvordan man skulle implmentere nye movestrategy'er der
indeholder meget af det samme logik vi har indført i de tidligere. Vi endte
med bare at copy-paste sammen fra 2 strategier, men hvis vi skulle have gjort
det uden at copy-paste, kunne vi have en instans af vores alphamonMoveStrategy
og refactore MoveCorrectDirectionMoveStrategy ud fra BetaMon og bruge den
strartegy i deltamon- og betamonMoveStrategy.
Der var ingen problemer med at indføre winnerStrategy.

\section{UML class diagram}
\includegraphics[width=410]{uml_class_3.eps}


\end{document}
