\documentclass[a4paper]{article}
\usepackage[latin1]{inputenc}
\usepackage[danish]{babel}
%\usepackage[T1]{fontenc}
\usepackage{amsmath,amssymb}

\begin{document}
\section{Board implementation}
\subsection{test list}
\begin{itemize}
\item Check om iteratoren indeholder pr\ae cis 28 ikke-null v\ae rdier.
\item Check der er pr\ae cis 30 checker spredt omkring p\aa\ boardet.
\item Check at color er red p\aa\ B1, black p\aa\ R1 og none p\aa\ B2.
\item flyt fra R6 til R5, check der er 1 checker p\aa\ R5 og 4 p\aa\ R6,
flyt fra R5 til R6, check der er 5 checkers p\aa\ R6 og 0 p\aa\ R5.

\item Check move R6 til R5, reset betyder at R5 count = 0 og R5 color=none
\end{itemize}

\subsection{Iterationer}

Vi implementerede testen: color er red p\aa\ B1, og s\aa\ den fejlede.
S\aa\ fakede vi den ved at altid lade getColor() returnere Color.RED.
Saa udvidede vi testen saa den ogs\aa\ checkede at B1 er sort.
Den failede, og vi triangulerede og indførte et map til at holde styr
p\aa\ farverne Map<Location, Color> og lod getColor retunere den korrekte
farve.
P\aa\ den m\aa de arbejede vi os igennem de ovenst\aa ende testcases



\section{Alphamon}

\subsection{test liste}

\begin{itemize}
\item dicerolls opfylder "The dice rolls are not random. The first two
  dice rolls give [1,2], the next two [3,4], next [5,6], and then it starts all over again."
\item newGame = playerInturn == Color.NONE

\item nextTurn() $\rightarrow$ ny spiller\\
indført Color currentPlayer, tilpasset nextTurn()


\item der er 30 brikker p\aa\ boardet\\
-- har flyttet en masse check metoder fra board over til game\\
-- har lavet en board varibel og brugt den direkte i getCount og getColor\\




\item f\o r move getNumberOf MovesLeft = 2


\item move R1 R3 $\rightarrow$ getNumberOf MovesLeft = 1


\item kun den der har tur m\aa\ flytte : 
  \begin{itemize}

  \item move R1 R3  true

  \item at R3 indeholder een brik der er sort

  \item move R6 R3 false 

  \end{itemize}


\item der er nogle p\aa\ et felt
  \begin{itemize}
  \item (move R1 R2 $\rightarrow$ true) 

  \item move R1 R6 $\rightarrow$ false

  \item og R1 indeholder 2 sorte brikker

  \end{itemize}






\item check diceValuesLeft: turn1, move R1, R3 $\rightarrow$ dicevaluesleft: 1

\item check numberOfMovesLeft: 
  \begin{itemize}
  \item turn1 $\rightarrow$ 2 numberofmovesleft,

  \item turn1, move(R1,R3) $\rightarrow$ 1 numberOfMovesLeft,

  \item turn1, move(R1,R3), move(R1,R2) $\rightarrow$ 0 numberOfMovesLeft.

  \end{itemize}


\item check getPlayerInTurnWithNoMoves: 
  \begin{itemize}
  \item turn1, move(R1,R3), move(R1,R2) $\rightarrow$ getPlayerInTurn() er Color.NONE.

  \end{itemize}
\item check winner:
  \begin{itemize}
  \item  turn1-5 $\rightarrow$ winner = none,

  \item turn6 $\rightarrow$ winner = red
  \end{itemize}


\item check moveWhenNoMoreMoves: turn1, move(R1,R3), move(R1,R2), move(R3,R4) $\rightarrow$ error


\end{itemize}

\section{Reflektioner}

I starten var det irriterende at lave test til det hele, men da man
skulle til at \ae ndre, refraktorer i koden vidste tdd sin styrke.


\end{document}
