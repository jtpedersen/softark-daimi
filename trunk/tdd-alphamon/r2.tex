\documentclass[a4paper]{article}
\usepackage[utf8]{inputenc}
\usepackage[danish]{babel}
%\usepackage[T1]{fontenc}
\usepackage{amsmath,amssymb}
\usepackage{fancyhdr}
\usepackage[dvipdfm]{graphicx}
\pagestyle{fancy}

\begin{document}
\chead{dSoftArk 2007 -- group 20 -- uge 1}
\lhead{}
\rhead{}


\section{Betamon Implementation}
\subsection{Refactoring}
Vi startede med at finde de ting i betamon-specifikationen der variere,
som var move-validering og terning-kast. Så vi implementerede et interface
til hver af disse og implementere med hensyn til strategy patternet. Vi
sørgede for at få alphamon til at virke med refactoringeringer før vi
begyndte at implementere betamon.
\subsection{Problems}
Vi diskuterede om hvor meget terning strategi interfaced skulle indeholder.
Mere specifikt, om den skulle indeholde alle data omkring terning-kast (
altser tilstand af terning-kast, og funktioner til at kaste, hente værdier,
og hente moves left. Eller om den bare skulle indeholde funktioner der
kaster terninger og fjerner terning-værdier.
Den første har den primære fordel at den incapsulere alt kode mht. til terning og
være helt fri for at lade Game tænke på det, hvormod den anden metode var
at terning-strategien ikke skulle vide noget om tilstanden, og pga. det
vil det være mere fleksibelt at ændre på strategien undervejs.
Vi blev aldrig enige om hvorvidt den ene eller anden var bedst, og endte med
at implementere metode nummer 2.

\section{Gammamon Implementation}
\subsection{Refactoring}
Der er ikke flere ting der variere i gammamon end i betamon, så vi indførte
ingen nye interfaces.
\subsection{Problems}
Det åbenlyse problem er at en tilfældig terning gør det meget dumt at 
inføre automatiske tests. Vi lavede testene først og endte med at lave
dumme tests der løkker indtil den har fundet alle værdier, og en anden
der løkker indtil den finder et dobbeltslag. Og da start-spilleren også
er tilfældigt valgt, indførte vi en masse if-statements der checker hvilken
spiller der startede, for at teste move-specifikationerne.
Vi tilføjede en sequence DieStrategy da vi så efter hvor dumme vi var,
så vi kunne havde simplificeret testene en del.

\section{UML class diagram}
\includegraphics[width=410]{uml_class_2.eps}

\section{UML sequence diagram}
\includegraphics[width=410]{uml_seq_2.eps}


\end{document}
