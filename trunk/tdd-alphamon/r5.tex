\documentclass[a4paper]{article}
\usepackage[utf8]{inputenc}
\usepackage[danish]{babel}
%\usepackage[T1]{fontenc}
\usepackage{amsmath,amssymb}
\usepackage{fancyhdr}
\usepackage{graphicx}
\pagestyle{fancy}

\begin{document}
\chead{dSoftArk 2007 -- group 20 -- uge 5}
\lhead{}
\rhead{}

\section{Use TDD to add the SemiMon variant to your portfolio of backgammon variants.}

Vi havde lavet isolerede tests, så det var primært integrationen vi
testede. Vi opdagde dog undervejs at vores dicestrategy havde fået for
meget at lave da den skulle fjerne terninger, og vi refactorede derfor
vores kode så standardgame kom til at håndtere det. Og skrev
selvfølgeligt først en testcase der viste vores bug, at de rigtige
terninger ikke blev fjernet.

\textit{Product: Full zip. Ant target "TestSemiMon".}

\section{ Make your SemiMon run with the GUI.}
It just works(tm)

\textit{Product: Full zip. Ant target "SemiMon".}

\section{Shesh-Besh and Acey-Deucy}

\textit{ There are several backgammon variants played on the
  backgammon board. Two of these are Shesh-Besh and Acey-Deucy that
  you can read about in the Backgammon variants page.}

\subsection{Sheesh-Besh}

Joker trækket kan håndteres af vores movestrategy, da den kan tjekke
games, diceThrown, og retunere værdier værdier ligesom i alphamon.

\subsection{Acey-Ducey}

Her kræves der lidt mere og for at implementere det ville vi foruden
en ny moveStrategy og to forskellige diceStrategies (manuelt valg og
random terninger), indføre et nyt interface: TurnHandler.

Turnhandleren skulle styre slagets gang i nextTurn, det vil sige at
den skulle vælge hvilken terningstrategi vi skulle bruge og finde ud
af hvilken spiller der var playerInTurn. Det kunne være at vi ville
blive nødt til at huske hvile moves der blev brugt i sidste omgang. 

Opstarten ville vi klare med at placere alle brikker på baren. Frem
for at indføre et nyt Board.


\newpage
\section{Polymorphic semimon}
\textit{Two implementation sketches: Use UML diagrams and pseudo-code 
sketches to outline your proposals.} \\

\subsection{extends betamon}


Men kan f.eks. extende fra BetaMon og så copy-paste de relevante
metoder fra Gamma- og Deltamon.


\begin{center}
\includegraphics[width=80mm]{fig2.eps}
\end{center} 

\begin{verbatim}
public class Seminmon extends Betamon {

 public Color winner() {
   ...
 }

 public void nextTurn() {
  ...
 }

}
\end{verbatim}



\newpage

\subsection{extends Alphamon}

Man kan f.eks. implementere det ved at arve fra AlphaMon, og så

\begin{center}
  \includegraphics[width=80mm]{fig1.eps}
\end{center}



\begin{verbatim}
public class Seminmon extends Alphamon {

 public Color winner() {
   ...
 }

 public void nextTurn() {
  ...
 }

 public boolean move(Location from, Location to) {
  ...
 }

}
\end{verbatim}


\section{Fordele og ulemper}
\textit{Contrast the polymorphic proposals for SemiMon with the
  compositional design: Analyze the benefits and liabilities of both
  the polymorphic and compositional proposals using terminology from
  the course literature.}


Det ville være pænest at nedarve direkte fra alphamon, da man så
ledes har semimon som en selvstændig entitet, men så ville man miste
fordelene ved nedarving, fordi så skal man implementere alle metoderne
igen.

Man kunne nedarve fra en af alle de nævnte klasser, og i alle tilfælde
vil man få et multiple maintaince problem. Derfor mener vi at den
kompositionelle tilgang er den bedste. 

\section{Backgammon}

Der er et ant target, ant backgammon og ant aimon hvis man vil spille
mod computeren.


\end{document}
