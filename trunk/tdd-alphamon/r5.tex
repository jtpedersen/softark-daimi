\documentclass[a4paper]{article}
\usepackage[utf8]{inputenc}
\usepackage[danish]{babel}
%\usepackage[T1]{fontenc}
\usepackage{amsmath,amssymb}
\usepackage{fancyhdr}
\usepackage{graphicx}
\pagestyle{fancy}

\begin{document}
\chead{dSoftArk 2007 -- group 20 -- uge 5}
\lhead{}
\rhead{}

\section*{4}
\textit{Two implementation sketches: Use UML diagrams and pseudo-code 
sketches to outline your proposals.} \\
Men kan f.eks. extende fra BetaMon og s!aa copy-paste de relevante
metoder fra Gamma- og Deltamon.
\begin{verbatim}

\end{verbatim}
\begin{center}
\includegraphics[width=80mm]{r5_fig2.eps}
\end{center} \\

Man kan f.eks. implementere det ved at arve fra AlphaMon, og s!aa

\includegraphics[width=80mm]{r5_fig1.eps}


\begin{verbatim}
   Figure f = objectServer.getEditor().drawing().
      findFigure(e.getX(), e.getY());
    if (f instanceof DieFigure) {
        objectServer.getGame().nextTurn();
    } 
\end{verbatim}






\end{document}
